\chapter{Białka zaangażowane w transport wapnia}\label{app:a}

\begin{scriptsize}
\begin{center}
\begin{longtable}{p{8cm}p{8cm}}
\caption[Białka sygnałosomu wapniowego w komórce]{Rozszerzona lista białek składających się na sygnałosom wapniowy w~komórkach eukaryotycznych~\cite{Berridge2012a}.}\label{tab:toolkit} \\
\toprule[0.12em]
\textbf{Białko} & \textbf{Opis}  \\
\midrule[0.06em]
\endfirsthead
\multicolumn{2}{c}%
{\tablename\ \thetable\ -- \textit{Lista białek biorących udział w tranporcie wapnia - kontynuacja\ldots}} \\
\midrule[0.06em]
\textbf{Białko} & \textbf{Opis}\\
\midrule[0.06em]
\endhead
\midrule[0.06em] \multicolumn{2}{r}{\textit{Kontynuacja na następnej stronie\ldots}} \\
\endfoot
\endlastfoot
\multicolumn{2}{l}{\textsc{\textbf{Recptory i przekaźniki}}} \\[0.175em]
\textbf{Receptory powiązane z białkami G (GPCRs)}& Receptory aktywujące PLC$\beta$\\[0.15em]
\textbf{Receptory powiązane z kinazą tyrozynową}& Receptory aktywujące PLC$\gamma$\\[0.1em]
\hspace{0.25cm} R. płytkopochodnego czynnika wzrostu (PDGFR)& Dostarczają miejsc dokujących do fosforylacji m.in. PLC$\gamma$\\
\hspace{0.75cm}PDGFR$\alpha$& \\
\hspace{0.75cm}PDGFR$\beta$& \\[0.1em]
\hspace{0.25cm} R. nabłonkowego czynnika wzrostu (EGFR)& \\
\hspace{0.75cm}ERBB1-ERBB4& \\[0.1em]
\hspace{0.25cm} R. naczyniowego czynnika wzrostu (VEGFR)& \\
\hspace{0.75cm}VEGFR1-VEGFR3& \\[0.15em]
\textbf{Białka G}& Białka adaptorowe dla receptora metabotropowego, hydrolizujące GTP do GDP+P \\[0.1em]
\hspace{0.25cm} G$\alpha$$_q$, G$\alpha$$_{11}$, G$\alpha$$_{14}$, G$\alpha$$_{15}$, G$\alpha$$_{16}$, G$\beta \gamma$& \\
\textbf{Czynnik wymiany nukleotydu guaninowego}& Uwalnia GDP z białek Ras\\[0.1em]
\hspace{0.25cm} RasGRF1& \\
\textbf{Regulatory białek G (RGS)}& Regulują aktywność podjednostki $\alpha$ białek G\\[0.1em]
\hspace{0.25cm} RGS1, RGS2, RGS4, RGS16&\\
\textbf{Fosfolipazy C (PLC)}& Katalizuje hydrolizę fosfatydów do DAG i IP$_3$\\[0.1em]
\hspace{0.25cm} PLC$\beta$1-4, PLC$\gamma$1-2, PLC$\delta$1-4, PLC$\varepsilon$, PLC$\zeta$& \\
\textbf{Cyklaza ADP-rybozy}& przekształca NAD$^+$ w cykliczną ADP-rybozę\\[0.1em]
\hspace{0.25cm} CD38& \\
\multicolumn{2}{l}{\textsc{\textbf{Kanały}}} \\[0.175em]
\textbf{Kanały błony komórkowej}& \\[0.15em]
\hspace{0.25cm} \textbf{Kanały bramkowane napięciem (VGCCs)}& \\[0.1em]
\hspace{0.75cm} Ca$_V$1.1 -- 1.4 (L-type)& \\
\hspace{0.75cm} Ca$_V$2.1 (P/Q-type)& \\
\hspace{0.75cm} Ca$_V$2.2 (N-type)& \\
\hspace{0.75cm} Ca$_V$2.3 (R-type)& \\
\hspace{0.75cm} Ca$_V$3.1 -- 3.3 (T-type)& \\
\hspace{0.25cm} \textbf{Kanały zależne od Ca$^{2+}$}& \\[0.1em]
\hspace{0.75cm} Aktywowane Ca$^{2+}$ kanały K$^+$& \\
\hspace{1cm} Kanały o niskiej przepuszczalności (SK)& \\
\hspace{1cm} Kanały o średniej przepuszczalnosci (IK)& \\
\hspace{1cm} Kanały o wysokiej przepuszczalnosci (BK)& \\
\hspace{0.75cm} Aktywowane Ca$^{2+}$ Kanały Cl$^-$& \\
\hspace{1cm} HCLCA1& \\[0.1em]
\hspace{0.25cm} \textbf{Kanały powiązane z receptorami (ROCs)}& \\
\hspace{0.75cm} Receptory nikotynowe & \\
\hspace{0.75cm} Receptory serotoninowe (5-HT$_3$)& \\
\hspace{0.75cm} Receptory AMPA & \\
\hspace{0.75cm} Receptory NMDA & \\
\hspace{0.75cm} Purynoreceptory (P2X) & \\
\hspace{0.25cm} \textbf{Kanały aktywowane przekaźnikami II-rzędu (SMOCs)}& \\
\hspace{0.75cm} I$_{ARC}$& \\
\hspace{0.25cm} \textbf{Kanały bramkowane nukleotydami cyklicznymi (CNGs)}& \\
\hspace{0.75cm} CNGA1-CNGA4, CNGB1, CNGB3& \\
\hspace{0.25cm} \textbf{Kanały  TRP}& Termo-wrażliwe receptory przejściowego potencjału \\
\hspace{0.75cm} TRPC1-TRPC7& \\
\hspace{0.75cm} TRPV1-TRPV6& \\
\hspace{0.75cm} TRPM1-TRPM8& \\
\hspace{0.25cm} Polycystins& \\
\hspace{0.75cm} PC1 -- PC2& \\
\textbf{Kanały uwalniające wapń z ER}& \\[0.15em]
\hspace{0.25cm} \textbf{Receptory InsP$_3$ (InsP3Rs)}& \\
\hspace{0.75cm} InsP$_3$R1 -- 3& \\
\hspace{0.25cm} \textbf{Receptory Ryanodynowe (RYRs)}& \\
\hspace{0.75cm} RYR1 -- 3& \\
\hspace{0.25cm} \textbf{Regulatory kanałów}& Glikoproteiny retikulum, regulujące aktywność kanałów IP$_3$R/RyR \\
\hspace{0.75cm} Triadyna& \\
\hspace{0.75cm} Junktyna& \\
\hspace{0.75cm} Sorcyna& \\
\hspace{0.75cm} FKBP12& \\
\hspace{0.75cm} FKBP12.6& \\
\hspace{0.75cm} Fosfolamban& \\
\textbf{Kanały uwalniające wapń z mitochondrium}& \\[0.15em]
\hspace{0.75cm} MICU1& \\
\hspace{0.75cm} Por permeabilizacyjny (PMT)& \\
\multicolumn{2}{l}{\textsc{\textbf{Bufory Ca$^{2+}$}}} \\[0.175em]
\textbf{Bufory cytozoliczne}& \\[0.15em]
\hspace{0.75cm} Kalbindyna D-28k& \\
\hspace{0.75cm} Kalretynina& \\
\hspace{0.75cm} Parwalbumina& \\
\textbf{Bufory ER}& \\[0.15em]
\hspace{0.75cm} Kalneksyna& \\
\hspace{0.75cm} Kalretikulina& \\
\hspace{0.75cm} Kalsekwestryna& \\
\hspace{0.75cm} GRP78& \\
\hspace{0.75cm} GRP94& \\
\multicolumn{2}{l}{\textsc{\textbf{Sensory Ca$^{2+}$}}} \\[0.175em]
\hspace{0.25cm} \textbf{Białka z domeną EF-hand}& \\[0.15em]
\hspace{0.75cm} Kalmodulina (CaM)& \\
\hspace{0.75cm} Kalcyneuryna B (CaNB)& \\
\hspace{0.75cm} Troponina C (TnC)& \\
\hspace{0.75cm} Miro& \\
\hspace{0.75cm} KInaza DAG$\alpha$& \\
\hspace{0.75cm} STIM& \\
\hspace{0.75cm} ALG-2& \\
\hspace{0.25cm} \textbf{Białka S100}& \\[0.15em]
\hspace{0.75cm} S100A1--14& \\
\hspace{0.75cm} S100B, S100C, S100P&\\
\hspace{0.25cm} \textbf{Neuronowe białka sensoryczne (NCS)}& \\[0.15em]
\hspace{0.75cm} NCS-1& \\
\hspace{0.75cm} Hippokalcyna& \\
\hspace{0.75cm} Neurokalcyna$\delta$ (TnC)& \\
\hspace{0.75cm} Rekoweryna& \\
\hspace{0.25cm} \textbf{Białka wizyno-podobne VILIPs)}& Białka wrażliwe na wapń, związują się z błoną komórkową pod wpływem wapnia \\[0.15em]
\hspace{0.75cm} VILIP-1--3& \\
\hspace{0.25cm} \textbf{Białka aktywujące cyklazą guanylylową (GCAPs)}& \\[0.15em]
\hspace{0.75cm} GCAP1--3& \\
\hspace{0.25cm} \textbf{Białka związane z kanałami Kv (KChIPs)}& \\[0.15em]
\hspace{0.75cm} KChIP1--4& \\
\hspace{0.25cm} \textbf{Białka wiążące Ca$^{2+}$ CaBPs)}& \\[0.15em]
\hspace{0.75cm} Kaldendryna& \\
\hspace{0.75cm} L-CaBP1& \\
\hspace{0.75cm} S-CaBP1& \\
\hspace{0.75cm} CaBP2--5& \\
\hspace{0.75cm} Kalneuron-1& \\
\hspace{0.75cm} Kalneuron-2& \\
\textbf{Białka z domeną C2}& \\[0.15em]
\hspace{0.25cm} \textbf{Synaptotagminy}& Sensory  Ca$^{2+}$ w błonach zakończeń aksonów\\[0.15em]
\hspace{0.75cm} Synaptotagmina I--III& \\
\hspace{0.25cm} Otoferyna& \\
\hspace{0.25cm} \textbf{Aneksyny}& Białko wiążące fosfolipidy podczas apoptozy\\
\hspace{0.75cm} Aneksyna A1--13& \\
\multicolumn{2}{l}{\textsc{\textbf{Enzymy i procesy zależne od  Ca$^{2+}$}}} \\[0.175em]
\textbf{Enzymy regulowane Ca$^{2+}$}& \\[0.15em]
\hspace{0.25cm} \textbf{Kinazy białkowe zależne od Ca$^{2+}$ (CaMKs)}& Fosforylazy Ser/Thr, aktywujące różne białka\\
\hspace{0.75cm} CaMKI--IV& \\
\hspace{0.75cm} MLCK& \\
\hspace{0.75cm} Kinaza fosforylazowa& \\
\hspace{0.75cm} Kinaza IP$_3$& \\
\hspace{0.75cm} Pyk2& \\
\hspace{0.25cm} \textbf{Kinazy lipidowe}&\\
\hspace{0.75cm} hVps34& \\
\hspace{1cm} PKC-$\alpha$,$\beta$I, $\beta$II, $\gamma$& \\
\hspace{0.25cm} \textbf{Fosfodiesterazy}&\\
\hspace{0.75cm} Fosfodiesterazy cyklicznego GMP (PDE)& \\
\hspace{1cm} PDE1A -- C& \\
\hspace{0.25cm} \textbf{Cyklazy adenylylowe (ACs)}&\\
\hspace{1cm} AC-1&\\
\hspace{1cm} AC-III&\\
\hspace{1cm} AC-VIII&\\
\hspace{1cm} AC-V&\\
\hspace{1cm} AC-VI& \\
\hspace{0.25cm} \textbf{Oksydazy (DUOX1-2)}&\\
\hspace{0.75cm} Syntaza NO (NOS)&\\
\hspace{1cm} Nabłonkowa NOS (eNOS)&\\
\hspace{1cm} Neuronowa NOS (nNOS)&\\
\hspace{0.25cm} \textbf{Aktywowane Ca$^{2+}$ proteazy}&\\
\hspace{1cm} Kalpanina I -- II&\\
\textbf{Czynniki transkrypcyjne}& \\[0.15em]
\hspace{0.25cm} \textbf{NFAT}& Zależny od Ca$^{2+}$ aktywator limfocytów T\\
\hspace{1cm} NFATc1--4&\\
\hspace{0.25cm} \textbf{CREB}& Regulator homeostazy glukozowej\\
\hspace{0.25cm} \textbf{DREAM}& Zaangażowany w przekaźnictwie bólu\\
\hspace{0.25cm} \textbf{CBP}&\\
\multicolumn{2}{l}{\textsc{\textbf{Pompy wapniowe}}} \\[0.175em]
\hspace{0.25cm} \textbf{Pompy Ca$^{2+}$ błony komórkowej (PMCAs)}&\\
\hspace{1cm} PMCA1--4&\\
\hspace{0.25cm} \textbf{Pompy Ca$^{2+}$ SER/ER (SERCAs)}&\\
\hspace{1cm} SERCA1--3&\\
\hspace{0.25cm} \textbf{Pompy Ca$^{2+}$ układu wydzielniczego (SPCA)}&\\
\hspace{1cm} SPCA1--2&\\
\multicolumn{2}{l}{\textsc{\textbf{Wymienniki}}} \\[0.175em]
\hspace{0.25cm} \textbf{Wymieniki Na$^+$/Ca$^{2+}$ (NCX)}&\\
\hspace{1cm} NCX1--3&\\
\hspace{0.25cm} \textbf{Wymieniki Na$^+$/Ca$^{2+}$/K$^+$ (NCKX)}&\\
\hspace{1cm} NCKX1--4&\\
\bottomrule[0.12em]
\end{longtable}
\end{center}
\end{scriptsize}