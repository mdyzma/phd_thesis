% -*- root: ../Thesis.tex -*-
\chapter{Podsumowanie}
\label{chap:podsumowanie}

Zasadniczym celem poniższej pracy jest analiza oscylacji stężeń jonów wapnia w komórce eukariotycznej. Dokładniej, przedmiotem naszego zainteresowania jest  czasowe zachowanie się stężeń niezwiązanych jonów wapnia w trzech kompartmentach komórki: cytozolu, retikulum cytoplazmatycznym oraz mitochondriach ze szczególnym uwzględnieniem obszarów bliskiego kontaktu pomiędzy retikulum endoplazmatycznym (ER) a mitochondriami. Obszary te określane są jako obszary MAM (od pierwszych liter określenia w języku angielskim: Mitochondria Associated Membrane Complexes).


Podstawą całokompartmentowych modeli oscylacji stężeń niezbuforowanych jonów wapnia jest założenie, że stężenia te wyrównują się dostatecznie szybko w ramach danego kompartmentu, tak że ich przestrzenne rozkłady w danej chwili czasu można zastąpić przez rozkłady uśrednione (po danym kompartmencie). Jest to założenie upraszczające, jednak bardzo często przyjmowane w literaturze  \cite{Atri1993,Borghans1997,Goldbeter1990,Marhl2000,Sneyd1995}. 
Dla kompartmentu cytozolicznego można je częściowo uzasadnić przyjmując, że efektywny współczynnik dyfuzji jonów wapniowych jest dostatecznie duży. W przypadku dwu pozostałych kompartmentów sytuacja jest bardziej skomplikowana, gdyż tworzą one skomplikowane struktury geometryczne, nierzadko podzielone na rozseparowane podstruktury, pomiędzy którymi nie może następować swobodna wymiana jonów wapnia. (Dodatkową komplikacją, jaka pojawia się przy rozpatrywaniu struktur typu MAM jest fakt, że są one rozłożone niejednorodnie w obszarze komórki.) Można jednak założyć, że dla interesujących nas rozwiązań oscylacyjnych  wyrównywanie się stężeń w~retikulum cytoplazmatycznym i mitochondriach dokonuje się na skutek ich lokalnego dostosowania się do jednorodnego przestrzennie stężenia jonów wapniowych w cytozolu. Dlatego też, pozostając na gruncie klasycznych modeli kompartmentalnych, zakładamy, że wewnątrz danego kompartmentu stężenie jonów wapniowych jest jednorodne przestrzennie. 

Punktem startowym do przedstawionych w Rozdziale~\ref{chap:model} \textbf{Modelu \#1} oraz \textbf{Modelu \#2} jest trójkompartmentowy model dynamiki jonów wapnia z pracy \cite{Marhl2000}. Modele te, zadane poprzez układy równań różniczkowych zwyczajnych, abstrahują od przestrzennego rozkładu struktur typu MAM i uwzględniają je poprzez wprowadzenie bezpośrednich przepływów pomiędzy mitochondrium a retikulum cytoplazmatycznym. Wprowadzenie powyższych przepływów nie wynika jedynie z niemożności uwzględnienia geometrii rozmieszczenia obszarów typu MAM, ale ma również swoje podłoże biologiczne. Każdy z nich stanowi bowiem sam w sobie strukturę niemalże zamkniętą, z~której wypływ jonów wapniowych jest istotnie utrudniony.


\medskip 

W Rozdziale~\ref{chap:wstep} opisane zostały komórkowe mechanizmy kontrolujące stężenia jonów wapniowych w poszczególnych kompartmentach, ze szczególnym uwzględnieniem kanałów, pomp i wymienników wapniowych w~retikulum endoplazmatycznym i w mitochondriach. Opisana została również struktura kompleksów mitochondrialno-retikularnych MAM. Obszary te odkryto już w latach 70-tych, jednak dopiero od niedawna są one przedmiotem intensywnych badań eksperymentalnych. Badania za pomocą mikroskopii elektronowej, FRET i metod genetycznych pozwoliły na dokładne określenie struktury fizycznej oraz zidentyfikowanie komponent tworzących MAM. Odległość między błonami odgraniczającymi w kompleksie waha się od 10 - 30 nm. Tworzą one zatem fizyczne połączenia przypominające synapsy umożliwiające szybsze przekazywanie jonów wapnia z ER do mitochondriów i odwrotnie. Interfejs mitochondrialno-retikularny stabilizowany jest przez szereg protein, które w większości powiązane są z głównymi elementami przewodzącymi sygnał wapniowy w tych kompartmentach, tj. receptorem IP$_3$R, pompą wapniową SERCA oraz kanałem VDAC. Szacuje się, że w~80\% przepływu jonów wapniowych pomiędzy mitochondrium a retikulum, odbywający się pośrednio przez cytozol, zachodzi poprzez obszary typu MAM \cite{Csordas1999,Mannella1998}. 

\medskip 

W Rozdziale~\ref{chap:modelowanie:osc}~ startując z matematycznego opisu czaso-przestrzennej dynamiki jonów wapnia w komórce wyrażonego w języku równań typu reakcji-dyfuzji na stężenia wolnych i zbuforowanych jonów wapnia, pokazujemy, jak przy założeniu odpowiednio dużych wartości współczynników dyfuzji, możemy otrzymać (asymptotycznie) układy równań różniczkowych zwyczajnych opisujących modele kompartmentowe. Pokazujemy też, jak przy założeniu dostatecznie szybkich reakcji przyłączania i odłączania jonów wapnia do miejsc buforujących w danym kompartmencie, równania różniczkowe na stężenia protein buforujących możemy aproksymować równaniami algebraicznymi, upraszczając tym samym istotnie matematyczną i numeryczna analizę układu. Tego rodzaju przybliżenia zostały zastosowane w zaproponowanych przez nas w Rozdziale 4 dwóch modelach kompartmentalnych mających na celu zbadanie wpływu obszarów typu MAM na oscylacje stężenia wapnia w komórce eukariotycznej. W rozdziale tym przedstawiamy również inny, bardzo istotny z fizjologicznego punktu widzenia, rodzaj czaso-przestrzennej dynamiki wapnia w komórkach - fale wapniowe. Sztandarowym przykładem tego rodzaju dynamiki są fale biegnące  wapnia w komórkach mięśniowych. W Podrozdziale~\ref{s:faleBiegnace} przedstawiamy rezultaty pracy \cite{Kazmierczak2013,Kazmierczak2010,Kazmierczak2011}, dotyczącej fal biegnących wapnia w długich komórkach. Praca ta bierze również pod uwagę zjawisko generowania naprężeń mechanicznych w komórce poprzez lokalne odchylenia stężenia wapnia od stanu równowagowego, jak również zjawisko odwrotne, tzn. pozyskiwanie wolnych jonów wapnia z magazynów retikularno-mitochondrialnych wskutek naprężeń mechanicznych. 


W Rozdz.~\ref{chap:modelowaniePrzestrzenne}  rozważamy niektóre aspekty przestrzennego modelowania dynamiki wapnia w komórce w oparciu o równania typu reakcji-dyfuzji, analizując m.in. problem korespondencji modeli rozszerzonych przestrzennie (opisywanych równani różniczkowymi cząstkowymi) oraz modeli ,,całokompartmentowych'', na uśrednione po kompartmentach stężenia jonów wapniowych.  


Modele przedstawione w Rozdz.~\ref{chap:model} opisane zostały w pracach \cite{Dyzma2012,Szopa2013}, które opisują dwa modele ewolucji uśrednionego stężenia jonów wapnia w trójkompartmentowym opisie komórki eukariotycznej. Jak już wspominaliśmy, celem obydwu modeli jest uwzględnienie połączeń między siateczką śródplazmatyczną a~mitochondriami, których istnienie pozwala na prawie bezpośredni przepływ jonów wapnia pomiędzy tymi organellami. Punktem wyjścia do budowy tych modeli jest model przedstawiony w pracy \cite{Marhl2000}. W Modelu \#1 modyfikacje w stosunku do modelu z pracy Marhla polegają na wprowadzeniu bezpośrednich przepływów mitochondrialno-retikularnych i zmianie niektórych parametrów układu. Wprowadzenie powyższych przepływów regularyzuje dynamikę układu: dla dostatecznie dużych wartości współczynnika $k_{MAM}$, określającego natężenie przepływów jonów wapnia poprzez interfejsy MAM, quasi-periodyczne rozwiązania chaotyczne, będące charakterystyczną cechą modelu Marhla, przestają istnieć i ustępują miejsca oscylacjom typu ,,bursting'', bądź też oscylacjom regularnym. Bezpośrednie przepływy retikularno-mitochondrialne zmieniają również efektywny okres oscylacji wapniowych. Okazuje się, że okres ten rośnie ze wzrostem wartości współczynnika $k_{MAM}$. Wprowadzenie przepływów retikularno-mitochondrialnych zmienia również jakościowo koegzystencję różnych typów rozwiązań układu. W przeciwieństwie do układu z pracy Marhla, oprócz stabilnych rozwiązań oscylacyjnych, układ posiada też trzy rozwiązania  stacjonarne, z których jedno jest stabilne. Wzrost wartości parametru $k_{MAM}$ powoduje zwiększenie się wielkości basenu przyciągania wspomnianego rozwiązania stacjonarnego, przy jednoczesnym zmniejszaniu się basenu przyciągania stabilnego rozwiązania periodycznego. W rezultacie, dla bardzo dużych wartości $k_{MAM}$ prawie wszystkie trajektorie układu są przyciągane przez stabilne rozwiązanie stacjonarne. Rozwiązanie to charakteryzuje się bardzo dużą wartością stężenia wolnych jonów wapnia w mitochondrium. Z drugiej strony wiadomo, że w początkowym stadium apoptozy komórek stężenia wapnia w mitochondrium istotnie wzrasta \cite{Cali2012,Giorgi2012,Rasola2011}. Tak więc, na gruncie przyjętego modelu, zjawiska prowadzące do zwiększenia efektywnego przepływu przez interfejs retikularno-mitochondrialny mogą prowadzić do apoptozy komórki. 
 

W Modelu \#2 zmodyfikowaliśmy dodatkowo wyrażenie opisujące wypływ jonów wapniowych z mitochondrium do postaci niezależnej od stężenia jonów wapniowych w cytozolu. Rozpatrzyliśmy też dwa możliwe mody pracy białka transportującego wapń do wnętrza mitochondrium - uniportera mitochondrialnego MICU: szybki mod RaM-owy (rapid uptake mode) realizujący się dla dostatecznie niskich wartości stężenia wapnia cytozolicznego oraz mod normalny. Obie te zmiany wpłynęły istotnie na strukturę rozwiązań rozpatrywanego układu. Dla wszystkich wartości $k_{MAM}$ w układzie mamy przynajmniej jedno rozwiązanie stabilne. Dla $k_{MAM}$ nie przekraczającego pewnej wartości progowej $k_{MAMP}$ jest to stabilne rozwiązanie periodyczne (stabilny cykl graniczny). Warto zauważyć, że w istocie rzeczy w Modelu \#2 mamy do czynienia z dwoma stabilnymi cyklami granicznymi istniejącymi dla różnych wartości $k_{MAM}$, których okres wzrasta skokowo  w punkcie bifurkacji wraz ze wzrostem $k_{MAM}$. Dla $k_{MAM} > k_{MAMP}$ wszystkie trajektorie układu są natomiast przyciągane przez stabilny punkt stacjonarny charakteryzujący się relatywnie dużą wartością stężenia wapnia w~mitochondrium. Jak w Modelu \#1 możemy zatem mówić, że proces inicjacji apoptozy komórki może być powiązany ze zwiększaniem się intensywności przepływów poprzez interfejsy MAM. 

\medskip 

W Rozdziale~\ref{chap:wyniki} podajemy eksperymentalne argumenty przemawiające za przedstawioną powyżej ideą związku początkowego stadium apoptozy z drastycznym zwiększeniem przepływu wapnia przez obszary typu MAM na skutek rożnego rodzaju czynników stresogennych.

\chapter{Realizacja celów pracy}
\label{chap:realizacjaCelow}
\begin{enumerate}
	\item \textbf{Cel szczegółowy 1.}
	
	\textbf{W Rozdz. 4 zaproponowaliśmy 2 modeli dynamiki wapnia w komórce eukariotycznej uwzględniające obszary bliskiego sąsiedztwa pomiędzy retikulum endoplazmatycznym a mitochondriami. Modele te zadane są poprzez równania różniczkowe zwyczajne na uśrednione po kompartymentach stężenia swobodnych (i zbuforowanych) jonów wapnia w komórce.  }
	
	\item \textbf{Cel szczegółowy 2.}
	
	\textbf{W oparciu o symulacje numeryczne zbadaliśmy wpływ parametru $k_{MAM}$, skalującego wielkość bezpośrednich przepływów retikularno - mitochndrialnych w kompleksach MAM, na istnienie, charakter i okres oscylacji stężeń jonów wapnia w poszczególnych kompartmentach.  Wykazaliśmy m.in.,  że bezpośredni przepływ wapnia powoduje wzrost okresu oscylacji oraz wywiera wpływ regularyzujący na czasowe profile oscylacji: quasi-periodyczne oscylacje chaotyczne znikają nawet dla niezbyt dużych wartości parametru $k_{MAM}$, a profile rozwiązań periodycznych stają się bardziej regularne.  Dla dostatecznie dużych wartości parametru $k_{MAM}$, oprócz stabilnych rozwiązań oscylacyjnych, układ posiada też trzy rozwiązania stacjonarne, z których jedno jest stabilne. Wzrost wartości parametru $k_{MAM}$ powoduje, że większość trajektorii układu znajduje się w obszarze przyciągania wspomnianego rozwiązania stacjonarnego.  Rozwiązanie to charakteryzuje się bardzo dużą wartością stężenia wolnych jonów wapnia w mitochondrium. Wiadomo, że w początkowym stadium apoptozy komórek stężenia wapnia w mitochondrium istotnie wzrasta. Tak więc, na gruncie przyjętego modelu, zjawiska prowadzące do zwiększenia efektywnego przepływu przez interfejs retikularno - mitochondrialny mogą prowadzić do apoptozy komórki.}
\end{enumerate}