\chapter{Realizacja celów pracy}

%Zaproponowane przez nas modele (oznaczone jako Model \#1 i Model \#2) oscylacji stężeń jonów wapnia uwzględniają jawnie miejsca bliskiego kontaktu mitochondriów i~ER - tzw. struktury MAM. Badania numeryczne wykazały, że   powyższe struktury mają istotny wpływ na rodzaj i kształt profili przebiegów czasowych oscylacji stężeń jonów wapniowych. W \textbf{Modelu \#1} wprowadziliśmy bezpośredni przepływ mitochondrialno-retikularny i zmieniliśmy niektóre parametry układu. Wprowadzenie powyższych przepływów regularyzuje dynamikę układu dla dostatecznie dużych wartości współczynnika $k_{MAM}$, określającego natężenie przepływów jonów wapnia poprzez interfejsy MAM. W \textbf{Modelu \#2} zmodyfikowaliśmy dodatkowo wyrażenie opisujące wypływ jonów wapniowych z mitochondrium do postaci niezależnej od stężenia jonów wapniowych w cytozolu. Rozpatrzyliśmy też dwa możliwe mody pracy białka transportującego wapń do wnętrza mitochondrium - uniportera mitochondrialnego MICU: szybki mod RaM-owy (\textbf{Ra}pid Uptake \textbf{M}ode) realizujący się dla dostatecznie niskich wartości stężenia wapnia cytozolicznego oraz mod normalny. Obie te zmiany wpłynęły istotnie na strukturę rozwiązań rozpatrywanego układu.
%
%\bigskip
%
%\textbf{W Modelu \#1:}
%\begin{enumerate}
%	\item  Wraz ze wzrostem przepływu jonów wapnia przez miejsca kontaktu uzyskano szereg typów oscylacyjnych, od prostych oscylacji stężeń jonów wapnia, przez złożone oscylacje typu ,,bursting'', po quasi-periodyczne rozwiązania chaotyczne. Zmiany parametru $k_{MAM}$ (czyli maksymalnej przepustowości interfejsu MAM) zmienia typ oscylacji, regularyzując je.
%	\item Amplituda i okres oscylacji oraz średnie stężenie jonów wapnia w~mitochondriach ($Ca_{Mit}$) rośnie z wielkością parametru $k_{MAM}$.
%	\item Wzrost współczynnika $k_{MAM}$ wpływa na kształt profili przebiegów czasowych, tak że dla wysokich wartości $k_{MAM}$ ładowanie mitochondriów wapniem zaczyna się o wiele szybciej, niż w przypadku niskich wartości $k_{MAM}$.
%	\item Dla dużych wartości $k_{MAM}$ basen przyciągania rozwiązań oscylacyjnych zmniejsza się znacząco i większość trajektorii zbiega do stabilnego punktu stacjonarnego $P_1$ (Ryc.~\ref{fig:basinofattractionMo1}), charakteryzującego się wysokim poziomem stężenia jonów wapnia w~mitochondriach, co może być interpretowane jako początkowa faza procesu apoptozy.
%\end{enumerate}
%
%\clearpage
%
%\textbf{W Modelu \#2:}
%\begin{enumerate}
%	\item Symulacje numeryczne wykazały istnienie wyłącznie oscylacji typu ,,bursting''.
%	\item Dla pewnych wartości parametru $K_{4,8}$ istnieją przedziały $k_{MAM}$, dla których współistnieją dwa cykle graniczne $LC_1$ i $LC_2$, różniące się amplitudą  i okresem oscylacji. Cykl graniczny $LC_1$ staje się stabilny  dla odpowiednio wysokich wartości parametru $k_{MAM}$. 
%	\item Dla odpowiednio wysokich wartości $k_{MAM}$ rozwiązania periodyczne zanikają, a~większość trajektorii przyciągana jest przez stabilny punkt stacjonarny $P$, charakteryzujący się wysokim poziomem jonów wapnia w mitochondriach (Ryc.~\ref{fig:diagram3_5}).
%\end{enumerate}



