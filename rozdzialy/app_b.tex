\chapter{Procentowy udział błon biologicznych w~komórce}\label{app:b}

\begin{table}[h]
\centering
\caption[Pole powierzchni organelli komórkowych]{Objętość hepatocytu = 5000$\mu$m$^3$, kom. trzustki = 1000$\mu$m$^3$. Całkowita powierzchnia błon lipidowych w hepatocycie = 110 000 $\mu$m$^2$, natomiast w kom. trzustki = 13 000 $\mu$m$^2$ \cite{Alberts2002}.}
\begin{tabular}{lcc}\toprule[0.12em]
\rule[-2ex]{0pt}{5.5ex}\textbf{Rodzaj błony lipidowej} & \multicolumn{2}{c}{\textbf{Procent całkowitej powierzchni błon}} \\\cline{2-3}
\rule[-2ex]{0pt}{5.5ex}& \textbf{Hepatocyt} & \textbf{Komórki trzustki} \\\midrule[0.06em]
Plazmalemma & 2 & 5 \\
RER & 35 & 60 \\
SER & 16 & <1 \\
Aparat Golgiego & 7 & 10 \\
Mitochondria &  &  \\
OMM & 7 & 4 \\
IMM & 32 & 17 \\
Jądro &  &  \\
Błona wewnętrzna jądra & 0.2 & 0.7 \\
Pęcherzyki wydzielnicze & NA & 3 \\
Lizosomy & 0.4 & NA \\
Peroksysomy & 0.4 & NA \\
Endosomy & 0.4 & NA \\\bottomrule[0.12em]
\end{tabular}
\end{table}