\chapter{Cele pracy}
\label{chap:cele}
\bigskip

%Celem pracy jest analiza wpływu obszarów bliskiego kontaktu pomiędzy retikulum endoplazmatycznym, a mitochondriami na homeostazę wapniową w komórce, poprzez zbadanie zachowań dwóch modeli matematycznych, w których uwzględniono nie tylko, wspomniane wcześniej, obszary bliskiego kontaktu, ale również właściwości białka transportującego wapń do wnętrza mitochondrium. A w szczególności:
%
%\bigskip
%
%\begin{enumerate}
%%\item Wykazaniu istnienia rozwiązań dla obu układów równań (Model \#1 i Model \#2).
%\item \textbf{Analiza  wpływu struktur MAM na przebiegi czasowe oscylacji stężeń jonów wapniowych w poszczególnych kompartmentach komórki.}
%\bigskip
%\item \textbf{Zbadanie wpływu struktur MAM na charakter oscylacji stężeń jonów wapnia (np. okres oscylacji, amplitudę oscylacji).}
%\bigskip
%%\begin{itemize}
%%	\item wykazanie istnienia rozwiązań periodycznych oraz wyznaczenie zakresów parametrów, dla których występują (Model \#1 i Model \#2)
%%	\item wykazanie istnienia  rozwiązań chaotycznych oraz wyznaczenie zakresów parametrów, dla których występują (Model \#1)
%%	\item zbadanie zależności okresu oscylacji od parametrów pisujących przepływ jonów wapnia przez MAM (Model \#1 i Model \#2)
%%	\item analiza zmian minimalnych i maksymalnych wartości stężeń jonów wapnia w kompartmencie cytozolicznym i mitochondrialnym w zależności od współczynnika $k_{MAM}$ (Model \#1)
%%\end{itemize}
%\item \textbf{Analiza zachowań rozwiązań obu układów równań.}
%\end{enumerate}


\textbf{CP: Celem niniejszej pracy jest zbadanie wpływu obszarów bliskiego sąsiedztwa pomiędzy retikulum endoplazmatycznym a mitochondriami na gospodarkę wapniową w komórkach eukariotycznych.}

\bigskip

W literaturze obszary takie często nazywane są kompleksami MAM (od pierwszych liter wyrazów nazwy w języku angielskim ,,mitochondria associated membranes''). Dokładny opis takich struktur wraz z referencjami literaturowymi  zamieszczony jest w~podrozdziale 1.7.1. Odległość między błonami ograniczającymi tych kompleksów wynosi około 9 nm dla siateczki śródplazmatycznej gładkiej oraz 30 nm w przypadku szorstkiej. Wewnątrz powyższych obszarów możliwy jest prawie bezpośredni przepływ jonów wapnia pomiędzy kompartmentem retikularnym a mitochondriami. Mimo, że kompleksy MAM przylegają jedynie do części zewnętrznych błon mitochondrialnych, badania eksperymentalne sugerują, że przepływ przez nie stanowi około 80\% całkowitego przepływu retikularno-mitochondrialnego. Wydaje się zatem, że znaczenie interfejsów typu MAM na międzykompartmentową dynamikę wapnia w komórce, w~szczególności na kształtowanie się stabilnych oscylacji wapniowych,  jest nie do przecenienia. Oscylacje stężeń swobodnych  jonów wapnia  są niezbędne do prawidłowego funkcjonowania komórki. Mogą być m.in. odpowiedzialne za szereg istotnych procesów fizjologicznych, takich jak kontrola cyklu komórkowego, skurcz mięśni szkieletowych, wzmocnienie synaptyczne. Stabilne oscylacje wapniowe stanowią również istotny czynnik będący częścią  sieci sygnałowej, sprawdzający, czy komórka jest w dobrej kondycji i utrzymujący ją przy życiu lub też indukujący apoptozę komórki. 

\bigskip

\textbf{Powyżej wymieniony ogólny cel pracy (CP)  zamierzamy wykonać w oparciu o realizację celów szczegółowych, którymi są:}

\begin{enumerate}
\item \textbf{Opracowanie modelu matematycznego opisującego efekty wynikające z istnienia kompleksów MAM.  Układ ten zadany jest  przez zwyczajne równania różniczkowe na zmienność w czasie uśrednionych po kompartmentach stężeń jonów wapniowych. }

\item \textbf{Zbadanie wpływu miejsc bliskiego kontaktu retikularno - mitochondrialnego na istnienie i charakter oscylacji stężeń jonów wapniowych w poszczególnych kompartmentach  komórki poprzez analizę zaproponowanych modeli matematycznych.  Tak więc, zainteresowani jesteśmy  w szczególności  zależnością  okresu  oscylacji i rodzaju wapniowych oraz ogólnej struktury rozwiązań rozpatrywanego układu od parametrów charakteryzujących intensywność przepływu przez kompleksy MAM. W ramach zaproponowanych modeli, chcielibyśmy również odpowiedzieć na pytanie: Czy istnienie kompleksów MAM, może wpływać również na długoczasowe zachowanie się komórki, np.  w warunkach stresu fizjologicznego?}
\end{enumerate}

\chapter{Struktura pracy}
\label{chap:struktura}
\bigskip

W  Rozdz. 1 opisane zostały  zjawiska i mechanizmy kontrolujące stężenia jonów wapniowych w poszczególnych kompartmentach komórek eukariotycznych. W szczególności opisana została  dokładnie struktura kompleksów mitochondrialno - retikularnych MAM. 

\bigskip

W Rozdz. 2 opisujemy  podstawy modelowania dynamiki wapnia w komórce w ujęciu nieprzestrzennym. Przedstawiamy m.in. matematyczne modele funkcjonowania poszczególnych kanałów, pomp i wymienników jonowych odpowiedzialnych za transport tych jonów ,,z'' i ,,do'' kompartmentów magazynujących wapń.  

\bigskip

W Rozdz. 3  przedstawiamy niektóre aspekty przestrzennego modelowania dynamiki wapnia w komórce w oparciu o równania typu reakcji-dyfuzji. W szczególności rozważamy problem korespondencji modeli rozszerzonych przestrzennie (opisywanych równani różniczkowymi cząstkowymi) oraz modeli ,,całokompartmentowych'' opisujących uśrednione po kompartmentach stężenia jonów wapniowych.  

\bigskip

W Rozdz. 4  zostały zaproponowane i przeanalizowane numerycznie dwa modele ewolucji uśrednionego stężenia jonów wapnia w trójkompartmentowym opisie komórki eukariotycznej. Jest to najistotniejsza część pracy, która, łącznie z Rozdz. 5, realizuje cele szczegółowe przedstawione powyżej. Prace kończymy Podsumowaniem (Rozdz.6).