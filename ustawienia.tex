\usepackage[utf8]{inputenc} % kodowanie polskich znaków
\usepackage[T1]{fontenc}

\usepackage{textcomp} %TRADEMARKI  I INNE SYMBOLE
\usepackage{verbatim,amsmath,amsfonts,amssymb,amsthm}
\usepackage[english,polish]{babel}
%
%
%%Wyglad pdf 
\usepackage{url}
\usepackage{color}
\usepackage{fix-cm}
\usepackage{setspace}  %odstepy miedzy wierszami
\usepackage{ae,aecompl}
\usepackage{lipsum} %generowanie losowego tekstu
\usepackage{titling}
\usepackage{xcolor}
\usepackage[version=3]{mhchem}
%\usepackage{pdflscape} %odwracanie stron w pdf
%\usepackage{lscape} %odwracanie stron
%%\usepackage{draftwatermark} %znak wodny [firstpage] ustawienia na koncu pliku


%...tabel
\usepackage{rotating} %obracabie tabel na stronie
\usepackage{array}
\usepackage{hhline}
\usepackage{multicol,multirow}
\usepackage{booktabs,bigdelim,bigstrut}
\usepackage{colortbl}

%...obrazkow
\usepackage{graphicx}
\usepackage{float}
\usepackage[section]{placeins} %kontroluje położenie float-ów
\usepackage{wrapfig}



%Formatowanie czcionek podpisow, tytułów rozdziałów etc
%\usepackage[font=footnotesize,labelfont=bf,format=default,justification=centerlast]{caption}
\usepackage{titlesec} %zmiana czcionek w tytule i sekcjach


%Listy, numerowania, numeracja wierszy
\usepackage{enumerate}
\usepackage{enumitem}
\usepackage{lineno}  % numerowanie wierszy
\usepackage{listings}
\usepackage{verbatim}

%Cytowania
%%\usepackage{cite}ków sto
\usepackage[nonamebreak]{natbib}


%% Rysowanie w LaTeXu
%\usepackage{pgfplots}
%\usepackage{pgf}
%\usepackage{tikz}
%\usetikzlibrary{arrows,positioning,shapes,calc,matrix}


% Słowniczek / lista skrótów
\usepackage[intoc]{nomencl}
%\makeglossary
\renewcommand{\nomname}{Wykaz skrótów}



% Styl punktowania

\newenvironment{bulletList}%
{ \begin{list}%
	{$\bullet$}%
	{\setlength{\labelwidth}{25pt}%
	 \setlength{\leftmargin}{30pt}%
	 \setlength{\itemsep}{\parsep}}}%
{ \end{list} }


% % % % % % % % % % % % % % % % % % % % % % % % % % % % % % % % % % % % % % %
% % % % % % % % % % % %% % % %  Nagłówki i stopki  % % % % % % % % % % % % %

% % % % % % % % % % % % % % % % % % % % % % % % % % % % % % % % % % % % % % % % % % % % %
% % % % % % % % % % % % % % % % %  Hiperlinki  % % % % % % % % % % % % % % % % % % % % %



% Links in pdf
\usepackage{color}
\definecolor{linkcol}{rgb}{0,0,0.4} 
\definecolor{citecol}{rgb}{0.5,0,0} 


\usepackage[a4paper,hyperindex=true]{hyperref} %pagebackref

\hypersetup
{
bookmarksopen=true,
pdftitle="Poradnik narzędzia Bio-IT",
pdfauthor="Michał Dyzma", 
pdfsubject="Narzędzia Bio-IT", %subject of the document
%pdftoolbar=false, % toolbar hidden
pdfmenubar=true, %menubar shown
pdfhighlight=/O, %effect of clicking on a link
colorlinks=true, %couleurs sur les liens hypertextes
%pdfpagemode=None, %aucun mode de page
%pdfpagelayout=SinglePage, %ouverture en simple page
pdffitwindow=true, %pages ouvertes entierement dans toute la fenetre
linkcolor=linkcol, %couleur des liens hypertextes internes
citecolor=citecol, %couleur des liens pour les citations
urlcolor=linkcol %couleur des liens pour les url
}


%\usepackage{memhfixc}

% % % % % % % % % % % % % % % % % % % % % % % % % % % % % % % % % % % % % % % % % % % %
% % % % % % % % % % % %  Definicje  % % % % % % % % % % % % % % % % % % % % % % % % % %



\setcounter{secnumdepth}{3}
\setcounter{tocdepth}{3}

\usepackage[hmarginratio=1:1,margin=2cm]{geometry}


% % % % % % % % % % % % % % % % % % % % % % % % % % % % % % % % % % % % % % % % % % % %
% % % % % % % % % % % % % % % % % % % % % % % % % % % % % % % % % % % % % % % % % % % %

\titleformat{\section}{\large\bfseries\scshape}{\thesection}{1em}{} % ustawieniewielkosci czcionki w sekcjach i tytule

\clubpenalty=10000
\widowpenalty=10000
%\raggedbottom